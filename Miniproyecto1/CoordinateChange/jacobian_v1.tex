\documentclass[11pt]{article}

\usepackage{amsmath,amssymb,mathtools}
\usepackage{physics}
\usepackage{geometry}
\usepackage{bm}
\usepackage{graphicx}
\usepackage{hyperref}
\geometry{margin=1in}
\hypersetup{colorlinks=true,linkcolor=blue,citecolor=blue,urlcolor=blue}

\title{Why the Jacobian Determinant Appears in Coordinate Changes for Integration and Discretization}
\author{}
\date{}

\begin{document}
\maketitle

\section{Core Idea}
When changing variables \( x = F(\hat x) \) in an integral, the infinitesimal volume (area, length) element transforms as
\[
dV_x = \abs{\det J_F(\hat x)}\, dV_{\hat x},
\]
where \( J_F(\hat x) = \frac{\partial x}{\partial \hat x} \) is the Jacobian matrix. The determinant measures how the mapping locally scales (and possibly flips) volumes. This scaling must be included so that the \emph{total amount} (mass, probability, energy) represented by the integral is preserved under reparameterization.

\section{Continuous Change of Variables}
Let \( F : \hat\Omega \to \Omega \subset \mathbb{R}^n\) be a smooth bijection with smooth inverse. For an integrable function \( f \),
\[
\int_{\Omega} f(x)\, d x = \int_{\hat \Omega} f(F(\hat x))\, \abs{\det J_F(\hat x)}\, d\hat x.
\]
Derivation: locally linearize
\[
F(\hat x + h) \approx F(\hat x) + J_F(\hat x) h.
\]
A small parallelepiped in the \(\hat x\)-space with edges \(h_1,\dots,h_n\) maps to one with volume
\(
\abs{\det(J_F(\hat x))}\, \abs{\det[h_1,\dots,h_n]}.
\)
Thus the measure scales by \(\abs{\det J_F}\).

\section{Geometric Meaning}
\begin{itemize}
  \item \( \det J_F > 0 \): local orientation preserved, scaled.
  \item \( \det J_F < 0 \): orientation reversed (reflection), use absolute value in scalar integration.
  \item Magnitude \( \abs{\det J_F} \): ratio of deformed local $n$-volume to reference one.
\end{itemize}

\section{Differentials vs. Derivatives}
Two different transformations occur:
\begin{enumerate}
  \item \textbf{Measures}: \( d x = \abs{\det J_F}\, d\hat x \).
  \item \textbf{Derivatives}: By chain rule,
  \(
  \nabla_{\hat x}(f\circ F) = J_F{(\hat x)}^{\top} \nabla_x f(F(\hat x)).
  \)
\end{enumerate}
Confusing these leads to errors: the Jacobian determinant multiplies \emph{integration measure}, not raw function values or gradients (unless deriving weak forms that integrate gradient products, where both appear in structured ways).

\section{Discretization Context}
Many numerical methods (quadrature, finite elements, finite volumes, spectral methods) map each physical element \(K\) to a simple \emph{reference element} \(\hat K\) to reuse precomputed rules.

\subsection{Reference-to-Physical Element Mapping}
Let \( F_K: \hat K \to K \). Then
\[
\int_{K} f(x)\, dx = \int_{\hat K} f(F_K(\hat x))\, \abs{\det J_{F_K}(\hat x)}\, d\hat x.
\]

\subsection{Quadrature Transfer}
Given a quadrature rule on \(\hat K\):
\[
\int_{\hat K} g(\hat x)\, d\hat x \approx \sum_{q=1}^Q w_q\, g(\hat x_q),
\]
it induces on \(K\):
\[
\int_{K} f(x)\, dx \approx \sum_{q=1}^Q w_q\, f(F_K(\hat x_q))\, \abs{\det J_{F_K}(\hat x_q)}.
\]
Hence each quadrature weight is scaled by the Jacobian determinant evaluated at the quadrature node.

\subsection{Interpolation / Shape Functions}
Interpolation on \(K\):
\[
u_h(x) = \sum_i U_i\, \phi_i^K(x), \qquad \phi_i^K(x) = \hat \phi_i(\hat x) \text{ with } x=F_K(\hat x).
\]
Evaluation at quadrature points uses mapped coordinates. The \emph{values} of shape functions transform by composition; the \emph{integration of their products} over \(K\) introduces \(\abs{\det J_{F_K}}\).

\subsection{Mass Matrix Example}
Local mass matrix entry:
\[
M_{ij}^K = \int_K \phi_i^K(x)\, \phi_j^K(x)\, dx
= \int_{\hat K} \hat \phi_i(\hat x)\, \hat \phi_j(\hat x)\, \abs{\det J_{F_K}(\hat x)}\, d\hat x.
\]
Discrete quadrature:
\[
M_{ij}^K \approx \sum_q w_q\, \hat \phi_i(\hat x_q)\, \hat \phi_j(\hat x_q)\, \abs{\det J_{F_K}(\hat x_q)}.
\]

\subsection{Stiffness Matrix (Gradient Terms)}
If gradients appear:
\[
K_{ij}^K = \int_K \nabla \phi_i^K \cdot \nabla \phi_j^K\, dx.
\]
Transforming gradients:
\[
\nabla \phi_i^K(x) = J_{F_K}{(\hat x)}^{-T} \nabla_{\hat x} \hat \phi_i(\hat x),
\]
so
\[
K_{ij}^K
= \int_{\hat K} \left(J^{-T}\nabla_{\hat x}\hat \phi_i\right)\cdot
\left(J^{-T}\nabla_{\hat x}\hat \phi_j\right)\, \abs{\det J}\, d\hat x
= \int_{\hat K} \nabla_{\hat x}\hat \phi_i^{\top}\, (J^{-1}J^{-T})\, \nabla_{\hat x}\hat \phi_j\, \abs{\det J}\, d\hat x.
\]
Both the inverse Jacobian (derivatives) and the determinant (measure) appear.

\section{1D, 2D, 3D Examples}
\paragraph{1D:} \( x = a + (b-a)\xi,\ \xi\in[0,1]. \) Then \( dx = (b-a)\, d\xi \). Integral:
\(
\int_a^b f(x)\, dx = \int_0^1 f(a+(b-a)\xi)\, (b-a)\, d\xi.
\)

\paragraph{2D:} Polar: \( x = r\cos\theta,\ y = r\sin\theta\). Jacobian determinant \( r\). Area element \( dx\,dy = r\, dr\, d\theta\).

\paragraph{3D:} Spherical: \( (x,y,z)= (r\sin\phi\cos\theta,\, r\sin\phi\sin\theta,\, r\cos\phi)\). Jacobian determinant \( r^2 \sin\phi\). Volume element \( dx\,dy\,dz = r^2 \sin\phi\, dr\, d\phi\, d\theta\).

\section{Orientation and Absolute Value}
For scalar integrals use \(\abs{\det J}\). In differential forms and oriented integration, the sign encodes orientation; numerical integration of positive densities typically uses the absolute value.

\section{Discrete Perspective Summary}
\begin{enumerate}
  \item Map quadrature nodes: \( x_q = F(\hat x_q)\).
  \item Evaluate integrand in physical coordinates via reference representation.
  \item Multiply each quadrature weight by \( \abs{\det J_F(\hat x_q)}\).
  \item Sum to approximate the physical integral.
\end{enumerate}

\section{Key Takeaway}
The Jacobian determinant is the local scaling factor between reference and physical volume elements. Discrete schemes must include it to preserve integral consistency, ensure conservation properties, and maintain correct convergence behavior.

\section{Minimal Pseudocode}
\begin{verbatim}
for each element K:
    assemble = 0
    for each quadrature point q:
        x_q = map(F, xhat_q)
        detJ = det(J_F(xhat_q))
        assemble += w_q * f(x_q) * abs(detJ)
\end{verbatim}

\section{Conclusion}
Including \(\abs{\det J}\) is not optional: it enforces measure preservation under reparameterization, making both continuous integrals and their discrete approximations invariant under smooth coordinate changes.

\end{document}